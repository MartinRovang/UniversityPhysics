% --------------------------------------------------------------
%     DENNE MALEN ER LAGET AV MARTIN SORIA RØVANG
%     TIL BRUK FOR OPPGAVELØSNINGER OG RAPPPRTER
%     GITHUB: github.com/martinrovang
% --------------------------------------------------------------


\documentclass[10pt]{article}
\usepackage{amsmath,amsthm,amssymb}
\usepackage{float}
\usepackage[norsk]{babel}
\usepackage[table]{xcolor}
\usepackage{color}
\usepackage{graphicx}
\usepackage{listings}
\usepackage{natbib}
\usepackage[utf8]{inputenc}
\usepackage{imakeidx}
\usepackage[a4paper]{geometry}
\usepackage[myheadings]{fullpage}
\usepackage{fancyhdr}
\usepackage{lastpage}
\usepackage{graphicx, wrapfig, subcaption, setspace, booktabs}
\usepackage[T1]{fontenc}
\usepackage[font=small, labelfont=bf]{caption}
\usepackage{fourier}
\usepackage[protrusion=true, expansion=true]{microtype}
\usepackage{url, lipsum}
\usepackage{tgbonum}
\usepackage{hyperref}
\usepackage{xcolor}
\usepackage[most]{tcolorbox}
\usepackage{mathtools}
\usepackage[page]{totalcount}
\usepackage{lastpage}


\newcommand{\HRule}[1]{\rule{\linewidth}{#1}}
\onehalfspacing
\setcounter{tocdepth}{5}
\setcounter{secnumdepth}{5}
\newcommand{\vect}[1]{\boldsymbol{#1}}

\definecolor{codegreen}{rgb}{0,0.6,0}
\definecolor{codegray}{rgb}{0.5,0.5,0.5}
\definecolor{codepurple}{rgb}{0.58,0,0.82}
\definecolor{backcolour}{rgb}{0.95,0.95,0.92}
\definecolor{skyblue}{rgb}{0.950, 1, 1}

\lstdefinestyle{mystyle}{
    backgroundcolor=\color{backcolour},   
    commentstyle=\color{codegreen},
    keywordstyle=\color{magenta},
    numberstyle=\tiny\color{codegray},
    stringstyle=\color{codepurple},
    basicstyle=\footnotesize,
    breakatwhitespace=false,         
    breaklines=true,                 
    captionpos=b,                    
    keepspaces=true,                 
    numbers=left,                    
    numbersep=5pt,                  
    showspaces=false,                
    showstringspaces=false,
    showtabs=false,                  
    tabsize=2,
    frame=single,
    %keywordstyle=\color{blue},
    language=Python,
    backgroundcolor = \color{skyblue}
}
 
\lstset{style=mystyle}
\lstset{
    basicstyle=\footnotesize\ttfamily,
  identifierstyle=\bfseries\color{green!40!black},
  commentstyle=\itshape\color{purple!40!black},
  keywordstyle=\color{blue},
  stringstyle=\color{orange},
}

\newcommand{\N}{\mathbb{N}}
\newcommand{\Z}{\mathbb{Z}}
 
\newenvironment{theorem}[2][Theorem]{\begin{trivlist}
\item[\hskip \labelsep {\bfseries #1}\hskip \labelsep {\bfseries #2.}]}{\end{trivlist}}
\newenvironment{lemma}[2][Lemma]{\begin{trivlist}
\item[\hskip \labelsep {\bfseries #1}\hskip \labelsep {\bfseries #2.}]}{\end{trivlist}}
\newenvironment{exercise}[2][Exercise]{\begin{trivlist}
\item[\hskip \labelsep {\bfseries #1}\hskip \labelsep {\bfseries #2.}]}{\end{trivlist}}
\newenvironment{problem}[2][Problem]{\begin{trivlist}
\item[\hskip \labelsep {\bfseries #1}\hskip \labelsep {\bfseries #2.}]}{\end{trivlist}}
\newenvironment{question}[2][Question]{\begin{trivlist}
\item[\hskip \labelsep {\bfseries #1}\hskip \labelsep {\bfseries #2.}]}{\end{trivlist}}
\newenvironment{corollary}[2][Corollary]{\begin{trivlist}
\item[\hskip \labelsep {\bfseries #1}\hskip \labelsep {\bfseries #2.}]}{\end{trivlist}}

\newenvironment{solution}{\begin{proof}[Solution]}{\end{proof}}
    
\makeindex[columns=3, title=Alphabetical Index, intoc]


% --------------------------------------------------------------
%                         Headers and footers
% --------------------------------------------------------------
\fancyhf{}
\pagestyle{fancy}
\rhead{Martin Soria Røvang}
\lhead{STA-2003-Tidsrekker}
\rfoot{Side \thepage \, av \pageref{LastPage}}
\renewcommand{\headrulewidth}{0.3pt}
\usepackage{amssymb}
\usepackage{gensymb}
\usepackage{amsmath}

\begin{document}
% --------------------------------------------------------------
%                         FRONTPAGE
% --------------------------------------------------------------
{\fontfamily{cmr}\selectfont
\title{ \normalsize \textsc{}
		\\ [1.0cm] % How much upper margin
		%\HRule{0.5pt} \\
        \LARGE \textbf{\uppercase{Home Exam}
        \HRule{0.5pt} \\ [0.5cm]
        FYS-2010-Digital image processing
        %\HRule{2pt} \\ [0.5cm]
        \\
		\normalsize \today \vspace*{5\baselineskip}}
		}

        \date{}
\author{
		Martin Soria Røvang \\ 
        Universitetet i Tromsø \\}

% \begin{titlepage}
\clearpage\maketitle
\vspace{0.2\textheight}
{\centering
Contains \pageref{LastPage} \, pages, including frontpage.\par
}
\thispagestyle{empty}
% \end{titlepage}

\newpage
\tableofcontents
% --------------------------------------------------------------
%                         Start here
% --------------------------------------------------------------


\newpage

\section{Task}
\subsection{a}

Replace this text with your summary/reflection of ``The Secret to Raising Smart Kids".  Your total reflection should be roughly a page long.
 
\cite{alpaydin_2014}


\subsection{b}












% --------------------------------------------------------------
%     Reference og appendix
% --------------------------------------------------------------
\newpage
\section{Appendix}

\begin{figure}[hbt!]
    \begin{lstlisting}

        """
        TASK B
        """
        
        import numpy as np 
        import matplotlib.pyplot as plt
        import os
        from PIL import Image
        import cv2
        from scipy.signal import convolve2d
        
        # Load images
        filedir = os.path.dirname(__file__)
        imagedir = 'images/'
        filename = 'Fig_part_B.tif'
        file = os.path.join(filedir, imagedir, filename)
        
        # Read image into array
        image = plt.imread(file)
        
        def resize_image(image, newN, newM):
            """Resizing by taking only the second pixel (50% resizing)"""
            resize = np.zeros((newM, newN))
            row, col = resize.shape
            try:
                for j in range(col):
                    for i in range(row):
                        resize[i, j] = image[i*2, j*2]
            except:
                pass
        
            return resize
        
        
        # Get image size
        row, col = image.shape
        # Resize image 50%
        imagerez = resize_image(image, int(col*0.5), int(row*0.5))
        
        # Plotting
        fig, ax = plt.subplots(1,2)
        ax[0].imshow(imagerez, cmap = 'gray', interpolation="none")
        ax[0].set_title('Resized 50%')
        ax[1].imshow(image, cmap = 'gray', interpolation="none")
        ax[1].set_title('Original image')
        plt.tight_layout()
        plt.savefig('resized.pdf', bbox_inches = 'tight',
            pad_inches = 0)
        plt.show()
        
        def smoothing(image, boxsize = 3):
                """
                Avarage smoothing of 2D array
                boxsize -> size of boxkernal filter
                returns filtered 2D array
                """
                boxkernal = np.ones((boxsize, boxsize))/(boxsize**2)
                result = convolve2d(image, boxkernal, mode = 'same')
                return result.astype('uint8')
        
        
        # Use these to smooth before resizing.
        
        smooth1 = smoothing(image, boxsize = 3)
        smooth2 = smoothing(image, boxsize = 5)
        smooth3 = smoothing(image, boxsize = 10)
        
        smooth1 = resize_image(smooth1, int(col*0.5), int(row*0.5))
        smooth2 = resize_image(smooth2, int(col*0.5), int(row*0.5))
        smooth3 = resize_image(smooth3, int(col*0.5), int(row*0.5))
        
        # smooth1 = resize_image(image, int(col*0.5), int(row*0.5))
        # smooth2 = resize_image(image, int(col*0.5), int(row*0.5))
        # smooth3 = resize_image(image, int(col*0.5), int(row*0.5))
        
        # smooth1 = smoothing(imagerez1, boxsize = 3)
        # smooth2 = smoothing(imagerez2, boxsize = 5)
        # smooth3 = smoothing(imagerez3, boxsize = 10)
        
        fig, ax = plt.subplots(1,3)
        ax[0].imshow(smooth1, cmap = 'gray', interpolation="none", vmin = 0, vmax = 255)
        ax[0].set_title('Smoothing, kernel 3x3')
        ax[1].imshow(smooth2, cmap = 'gray', interpolation="none", vmin = 0, vmax = 255)
        ax[1].set_title('Smoothing, kernel 5x5')
        ax[2].imshow(smooth3, cmap = 'gray', interpolation="none", vmin = 0, vmax = 255)
        ax[2].set_title('Smoothing, kernel 10x10')
        plt.tight_layout()
        plt.savefig('smoothing.pdf', bbox_inches = 'tight', pad_inches = 0)
        plt.show()
        
        def lapsharp(image, maskret = False):
                """
                img -> 2D array
                maskret = True -> returns result and mask
                maskret = False -> returns result
                """
                #padded_image = np.pad(img, (1, 1), mode = 'symmetric')
                # lap is linear therefore;
                # lap f(x,y) = f(x + 1, y) + f(x - 1, y) + f(x, y + 1) + f(x, y - 1) - 4f(x,y)...
                #--------------------
                c = -1 # Depends on kernel
                # make zero kernal
                lapmask = np.zeros((3, 3))
                
                # add values to kernel
                lapmask[0,0] = 1
                lapmask[0,1] = 1
                lapmask[0,2] = 1
        
                lapmask[1,0] = 1
                lapmask[1,1] = -8
                lapmask[1,2] = 1
        
                lapmask[2,0] = 1
                lapmask[2,1] = 1
                lapmask[2,2] = 1
                #--------------------
                mask = convolve2d(image, lapmask, mode = 'same')
                result = image + c*mask
        
                # Map values to 0-255
                g1 = image - np.min(image)
                g = g1/np.max(g1) *255
                g = g.astype('uint8')
        
                if maskret == True:
                    return g, mask
                else:
                    return g.astype('uint8')
        
        def gaussian_hp(image, sigma):
                """Gaussian high pass filter sigma defines the radius around the centered frequency"""
                row, col = image.shape
                H = np.zeros((row, col))
                for y in range(row):
                    for x in range(col):
                        D = np.sqrt((y-int(row/2))**2 + (x-int(col/2))**2)
                        H[y,x] = np.exp(-D**2/(2*sigma**2))
                H_hp = (1-H)
                X = np.fft.fftshift(np.fft.fft2(image))
                Y = np.fft.fftshift((1+H_hp)*X)
                y = np.fft.ifft2(Y)
                return np.abs(y)
        
        sigma = 40
        
        # If enabled blurring before resizing
        # sharpedimage = lapsharp(smooth1)
        # sharpedimage2 = gaussian_hp(smooth1, sigma) + smooth1  # Adding highpass mask to blurred image to sharp it.
        
        
        sharpedimage = lapsharp(smooth1)
        sharpedimage2 = gaussian_hp(smooth1, sigma)  # Adding highpass mask to blurred image to sharp it.
        
        
        # Plotting
        fig, ax = plt.subplots(1,2)
        ax[0].imshow(sharpedimage, cmap = 'gray', vmin = 0, vmax = 255, interpolation="none")
        ax[0].set_title('Sharpened, Laplace kernel')
        ax[1].imshow(sharpedimage2, cmap = 'gray', vmin = 0, vmax = 255, interpolation="none")
        ax[1].set_title('Sharpened, gaussian highpass $\sigma = %s$'%sigma)
        plt.tight_layout()
        plt.savefig('sharpened.pdf', bbox_inches = 'tight',
            pad_inches = 0)
        plt.show()        
    \end{lstlisting}
\caption{TASK B}
\label{TASK B}
\end{figure}



\begin{figure}[hbt!]
    \begin{lstlisting}
        """
        TASK C1-3
        
        """
        import numpy as np 
        import matplotlib.pyplot as plt
        import os
        from PIL import Image
        import cv2
        from scipy.signal import convolve2d
        
        # Load images
        filedir = os.path.dirname(__file__)
        imagedir = 'images/'
        filename1 = 'F1.png'
        file1 = os.path.join(filedir, imagedir, filename1)
        
        filename2 = 'F2.png'
        file2 = os.path.join(filedir, imagedir, filename2)
        
        filename3 = 'F3.png'
        file3 = os.path.join(filedir, imagedir, filename3)
        
        filename4 = 'F4.png'
        file4 = os.path.join(filedir, imagedir, filename4)
        
        filename5 = 'F5.png'
        file5 = os.path.join(filedir, imagedir, filename5)
        
        F1 = plt.imread(file1)
        F2 = plt.imread(file2)
        F3 = plt.imread(file3)
        F4 = plt.imread(file4)
        F5 = plt.imread(file5)
        
        # Transform to 0-255
        F1 = (F1-np.min(F1))/np.max(F1-np.min(F1))*255
        F2 = (F2-np.min(F2))/np.max(F2-np.min(F2))*255
        F3 = (F3-np.min(F3))/np.max(F3-np.min(F3))*255
        F4 = (F4-np.min(F4))/np.max(F4-np.min(F4))*255
        F5 = (F5-np.min(F5))/np.max(F5-np.min(F5))*255
        F1 = F1.astype('uint8')
        F2 = F2.astype('uint8')
        F3 = F3.astype('uint8')
        F4 = F4.astype('uint8')
        F5 = F5.astype('uint8')
        
        # Make histogram
        fig, ax = plt.subplots(1,2)
        ax[0].hist(F1.flatten(), bins = 256)
        ax[0].set_title('Histogram F1')
        
        ax[1].hist(F2.flatten(), bins = 256)
        ax[1].set_title('Histogram F2')
        
        plt.savefig('noisehist.pdf')
        plt.tight_layout()
        plt.show()
        
        
        def dB(x):
            """dB scale of signal"""
            return 10*np.log10(x+1)
        
        
        fig, ax = plt.subplots(1,3)
        
        spectrum3 = np.fft.fftshift(np.fft.fft2(F3))
        spectrum4 = np.fft.fftshift(np.fft.fft2(F4))
        spectrum5 = np.fft.fftshift(np.fft.fft2(F5))
        ax[0].imshow(np.abs(spectrum3), cmap = 'seismic')
        ax[0].set_title('F3')
        ax[0].set_xlim([260,340])
        ax[0].set_ylim([260,340])
        ax[1].imshow(np.abs(spectrum4), cmap = 'seismic')
        ax[1].set_title('F4')
        ax[1].set_xlim([260,340])
        ax[1].set_ylim([260,340])
        ax[2].imshow(np.abs(spectrum5), cmap = 'seismic')
        ax[2].set_title('F5')
        ax[2].set_xlim([260,340])
        ax[2].set_ylim([260,340])
        plt.tight_layout()
        plt.savefig('fourierrep.pdf')
        plt.show()
        
        
        
        def medianfilter(image, boxsize = 3):
                """
                Uses median filter, convolution in spatial domain\n
                returns processed image.
                """
                image_padded = np.pad(image, (boxsize,boxsize) , mode = 'symmetric')
                result = np.zeros(image.shape)
                rows, cols = image.shape
                # Traverse the image and apply filter.
                for row in range(rows):
                    for col in range(cols):
                        result[row, col] = np.median(image_padded[row:row + boxsize,col:col + boxsize].flatten())
                return result
        
        # Filter image
        filteredsalt = medianfilter(F1)
        
        # Plot
        fig, ax = plt.subplots(1,2)
        ax[0].imshow(F1, cmap = 'gray', vmin = 0, vmax = 255, interpolation="none")
        ax[1].imshow(filteredsalt, cmap = 'gray', vmin = 0, vmax = 255, interpolation="none")
        ax[0].set_title('Original')
        ax[1].set_title('Salt/pepper image, median filtered.')
        plt.tight_layout()
        plt.savefig('Filtered_saltimage.pdf', bbox_inches = 'tight',
            pad_inches = 0)
        plt.show()
        

        def adaptive_filter(image, boxsize = 3):
                """
                Adptive filtering, found variance of noise by find variance of a slice in the top image.
                Uses local var and local mean to set pixel value.
                """
                # Pad image
                image_padded = np.pad(image, (boxsize, boxsize) , mode = 'symmetric')
                result = np.zeros(image.shape)
                rows, cols = image_padded.shape
                # Get histogram of strip in image
                hist, bins = np.histogram(F2[0:100,:].flatten(),256)
                # Intensities
                r = np.array([x for x in range(256)])
                # Get size
                rows, cols = image.shape
                # Get probabilites
                prob = hist/(rows*cols)
                # Get mean
                m = np.sum(r*prob)
                # Get variance estimate of noise
                variance_noise = np.sum((r-m)**2*prob)
                
                # Traverse the image and apply filter.
                for row in range(rows):
                    for col in range(cols):
                        local_var = np.var(image_padded[row:row + boxsize,col:col + boxsize])
                        local_mean = np.mean(image_padded[row:row + boxsize,col:col + boxsize])
                        kernel_placement = image_padded[row:row + boxsize,col:col + boxsize] 
                        current_val = image_padded[row,col]
                        if variance_noise > local_var:
                            result[row, col] = current_val - 1*(current_val - local_mean)
                        
                        else:
                            result[row, col] = current_val - (variance_noise/local_var)*(current_val - local_mean)
                
                return result.astype('uint8')
        
        # Run filter on image F2
        filteredimage = adaptive_filter(F2, boxsize= 3)
        
        # Plot
        fig, ax = plt.subplots(1,2)
        ax[0].imshow(F2, cmap = 'gray', interpolation = 'none', vmin = 0, vmax = 255)
        ax[1].imshow(filteredimage, cmap = 'gray', interpolation = 'none', vmin = 0, vmax = 255)
        ax[0].set_title('Original')
        ax[1].set_title('Adaptive filtered')
        plt.tight_layout()
        plt.savefig('gaussianfiltered.pdf', bbox_inches = 'tight',
            pad_inches = 0)
        plt.show()
    \end{lstlisting}
\caption{TASK C}
\label{TASK C}
\end{figure}


\begin{figure}[hbt!]
    \begin{lstlisting}
        """
        TASK C4
        
        """
        
        
        
        import numpy as np 
        import matplotlib.pyplot as plt
        import os
        from PIL import Image
        import cv2
        from scipy.signal import convolve2d
        
        # Load images
        filedir = os.path.dirname(__file__)
        imagedir = 'images/'
        filename1 = 'F1.png'
        file1 = os.path.join(filedir, imagedir, filename1)
        
        filename2 = 'F2.png'
        file2 = os.path.join(filedir, imagedir, filename2)
        
        filename3 = 'F3.png'
        file3 = os.path.join(filedir, imagedir, filename3)
        
        filename4 = 'F4.png'
        file4 = os.path.join(filedir, imagedir, filename4)
        
        filename5 = 'F5.png'
        file5 = os.path.join(filedir, imagedir, filename5)
        
        F1 = plt.imread(file1)
        F2 = plt.imread(file2)
        F3 = plt.imread(file3)
        F4 = plt.imread(file4)
        F5 = plt.imread(file5)
        
        # Transform to 0-255
        F1 = (F1-np.min(F1))/np.max(F1-np.min(F1))*255
        F2 = (F2-np.min(F2))/np.max(F2-np.min(F2))*255
        F3 = (F3-np.min(F3))/np.max(F3-np.min(F3))*255
        F4 = (F4-np.min(F4))/np.max(F4-np.min(F4))*255
        F5 = (F5-np.min(F5))/np.max(F5-np.min(F5))*255
        
        F1 = F1.astype('uint8')
        F2 = F2.astype('uint8')
        F3 = F3.astype('uint8')
        F4 = F4.astype('uint8')
        F5 = F5.astype('uint8')
        
        
        def gaussian_lp(image, sigma):
                """Gaussian low pass filter, sigma defines the radius around the centered frequency(cutoff)"""
                row, col = image.shape
                H = np.zeros((row, col))
                for y in range(row):
                    for x in range(col):
                        D = np.sqrt((y-int(row/2))**2 + (x-int(col/2))**2)
                        H[y,x] = np.exp(-D**2/(2*sigma**2))
                X = np.fft.fftshift(np.fft.fft2(image))
                Y = np.fft.fftshift(X*H)
                y = np.fft.ifft2(Y)
                return np.abs(y), H
        
        # Cutoff
        sigma = 40
        # Lowpass image F1
        image, H = gaussian_lp(F1, sigma)
        
        
        # Plot gaussian lowpass
        fig, ax = plt.subplots(1,1)
        ax.imshow(H, cmap = 'gray')
        ax.set_title('Gaussian lowpass in f-domain $\sigma$ = %s' %sigma)
        plt.savefig('gausf-dom.pdf', bbox_inches = 'tight',
            pad_inches = 0)
        plt.show()
        plt.show()
        
        
        
        # Plot
        fig, ax = plt.subplots(1,2)
        ax[0].imshow(F1, cmap = 'gray', interpolation = 'none', vmin = 0, vmax = 255)
        ax[0].set_title('Original F1')
        ax[1].imshow(image, cmap = 'gray', interpolation = 'none', vmin = 0, vmax = 255)
        ax[1].set_title('Gaussian lowpass, $\sigma$ = %s'%sigma)
        plt.tight_layout()
        plt.savefig('C41.pdf', bbox_inches = 'tight',
            pad_inches = 0)
        plt.show()
        
        # Lowpass image F2
        image, H = gaussian_lp(F2, sigma)
        
        # Plot
        fig, ax = plt.subplots(1,2)
        ax[0].imshow(F2, cmap = 'gray', interpolation = 'none', vmin = 0, vmax = 255)
        ax[0].set_title('Original F2')
        ax[1].imshow(image, cmap = 'gray', interpolation = 'none', vmin = 0, vmax = 255)
        ax[1].set_title('Gaussian lowpass, $\sigma$ = %s'%sigma)
        plt.tight_layout()
        plt.savefig('C42.pdf', bbox_inches = 'tight',
            pad_inches = 0)
        plt.show()
    \end{lstlisting}
\caption{TASK C4}
\label{TASK C4}
\end{figure}

\begin{figure}[hbt!]
    \begin{lstlisting}
        """
        TASK C5
        """
        
        import numpy as np 
        import matplotlib.pyplot as plt
        import os
        from PIL import Image
        import cv2
        from scipy.signal import convolve2d
        
        # Load images
        filedir = os.path.dirname(__file__)
        imagedir = 'images/'
        filename1 = 'F1.png'
        file1 = os.path.join(filedir, imagedir, filename1)
        
        filename2 = 'F2.png'
        file2 = os.path.join(filedir, imagedir, filename2)
        
        filename3 = 'F3.png'
        file3 = os.path.join(filedir, imagedir, filename3)
        
        filename4 = 'F4.png'
        file4 = os.path.join(filedir, imagedir, filename4)
        
        filename5 = 'F5.png'
        file5 = os.path.join(filedir, imagedir, filename5)
        
        F1 = plt.imread(file1)
        F2 = plt.imread(file2)
        F3 = plt.imread(file3)
        F4 = plt.imread(file4)
        F5 = plt.imread(file5)
        
        # Transform to 0-255
        F3 *= 255
        F4 *= 255
        F5 *= 255
        F3 = F3.astype('uint8')
        F4 = F4.astype('uint8')
        F5 = F5.astype('uint8')
        
        # Fourier, shift absolute value and decibel 
        Y2 = 10*np.log10(np.abs(np.fft.fftshift(np.fft.fft2(F4))))
        
        # Get frequencies sampling 1
        N2 , M2 = Y2.shape
        Y2freqN = np.fft.fftshift(np.fft.fftfreq(N2, 1))
        Y2freqM = np.fft.fftshift(np.fft.fftfreq(M2, 1))
        
        # plot
        fig, ax = plt.subplots(1,2)
        ax[0].imshow(F4, cmap='gray', interpolation = 'none', vmin = 0, vmax = 255)
        ax[0].set_title('F4')
        ax[1].imshow(Y2, cmap=plt.cm.BuPu_r, extent=(Y2freqN.min(),Y2freqN.max(),Y2freqM.min(),Y2freqM.max()))
        ax[1].set_title('Frequency spectrum in dB scale, T = 20')
        plt.savefig('findfreqfreq.pdf', bbox_inches = 'tight',
            pad_inches = 0)
        plt.show()
    \end{lstlisting}
\caption{TASK C6}
\label{TASK C6}
\end{figure}



\begin{figure}[hbt!]
    \begin{lstlisting}

        """
        TASK C6
        """
        
        import numpy as np 
        import matplotlib.pyplot as plt
        import os
        from PIL import Image
        import cv2
        from scipy.signal import convolve2d
        
        
        
        # Loading files
        filedir = os.path.dirname(__file__)
        imagedir = 'images/'
        filename5 = 'F5.png'
        file5 = os.path.join(filedir, imagedir, filename5)
        
        # Image import as floating values convert to 255
        F5 = plt.imread(file5)
        F5 *= 255
        F5 = F5.astype('uint8')
        
        # Fourier transform
        spectrum = np.fft.fftshift(np.fft.fft2(F5))
        spectrum1 = 10*np.log10(np.abs(spectrum))
        
        # get shape
        N , M = spectrum1.shape
        # Find right frequency, sampling rate 1
        HfreqN = np.fft.fftshift(np.fft.fftfreq(N, 1))
        HfreqM = np.fft.fftshift(np.fft.fftfreq(M, 1))
        
        
        # Plot
        fig, ax = plt.subplots(1,2)
        ax[0].imshow(F5, cmap = 'gray', interpolation = 'none', vmin = 0, vmax = 255)
        ax[1].set_title('Spectrum F5')
        ax[1].imshow(spectrum1, cmap = plt.cm.BuPu_r, aspect='auto', extent=(HfreqN.min(),HfreqN.max(),HfreqM.min(),HfreqM.max()))
        ax[0].set_title('F5')
        # ax[1].set_xlim([-1, 1])
        # ax[1].set_ylim([-1, 1])
        plt.savefig('superpositionfreq.pdf', bbox_inches = 'tight',
            pad_inches = 0)
        plt.show()
    \end{lstlisting}
\caption{TASK C6}
\label{TASK C6}
\end{figure}



\begin{figure}[hbt!]
    \begin{lstlisting}
        """TASK C7-1"""


        import numpy as np 
        import matplotlib.pyplot as plt
        import os
        from PIL import Image
        import cv2
        from scipy.signal import convolve2d
        
        
        
        # load files
        filedir = os.path.dirname(__file__)
        imagedir = 'images/'
        filename5 = 'F5.png'
        file5 = os.path.join(filedir, imagedir, filename5)
        
        filename3 = 'F3.png'
        file3 = os.path.join(filedir, imagedir, filename3)
        
        filename4 = 'F4.png'
        file4 = os.path.join(filedir, imagedir, filename4)
        
        filename5 = 'F5.png'
        file5 = os.path.join(filedir, imagedir, filename5)
        
        # Read images
        F3 = plt.imread(file3)
        F4 = plt.imread(file4)
        F5 = plt.imread(file5)
        
        
        # Transform to 0-255
        F3 = (F3-np.min(F3))/np.max(F3-np.min(F3))*255
        F4 = (F4-np.min(F4))/np.max(F4-np.min(F4))*255
        F5 = (F5-np.min(F5))/np.max(F5-np.min(F5))*255
        F3 = F3.astype('uint8')
        F4 = F4.astype('uint8')
        F5 = F5.astype('uint8')
        
        
        def butter_notch_filter(image, sigma, notches, n):
                """Butterworth notch reject, sigma defines the radius around the notch frequency
                and n defines the order.(how close to ideal you want)"""
                row, col = image.shape
                res = np.ones((row, col))
                for h in notches:
                    H = np.zeros((row, col))
                    for y in range(row):
                        for x in range(col):
                            Dp = np.sqrt((y-int(row/2)-h[1])**2 + (x-int(col/2)-h[0])**2)
                            Dn = np.sqrt((y-int(row/2)+h[1])**2 + (x-int(col/2)+h[0])**2)
                            if Dp > 0 and Dn > 0:
                                H[y,x] = (1/(1+ (sigma/Dp)**n))*(1/(1+ (sigma/Dn)**n))
                            else:
                                H[y,x] = 0
                    res *= H
                H = res
                X = np.fft.fftshift(np.fft.fft2(image))
                Y = np.fft.fftshift(X*H)
                y = np.fft.ifft2(Y)
                return np.abs(y), H
        
        def butterworth_hp_sharp(image, sigma, n, k):
                """Butterworth high pass filter, sigma defines the radius around the centered frequency
                and n defines the order.(how close to ideal you want)"""
                row, col = image.shape
                H = np.zeros((row, col))
                for y in range(row):
                    for x in range(col):
                        D = np.sqrt((y-int(row/2))**2 + (x-int(col/2))**2)
                        if D > 0:
                            H[y,x] = 1/(1+ (sigma/D)**(2*n))
                        else:
                            H[y,x] = 0
                X = np.fft.fftshift(np.fft.fft2(image))
                # sharp image (High frequency emphasis filtering)
                Y = (1+k*H)*X
                Y = np.fft.fftshift(Y)
                y = np.fft.ifft2(Y)
                return np.abs(y)
        
        # Centered frequency, since notch filter have origo in center we need this to subtract.
        mid = 300
        # Notches to reject
        notches = np.array([[0, 330-mid]])
        
        
        # Use filters
        y, H = butter_notch_filter(F3, 1, notches, 3)
        y = butterworth_hp_sharp(y, 8, 5, k = 1)
        
        # transform values 0-255
        y = (y-np.min(y))/np.max(y-np.min(y))*255
        y = y.astype('uint8')
        
        # Get frequencies
        N , M = H.shape
        HfreqN = np.fft.fftshift(np.fft.fftfreq(N, 1))
        HfreqM = np.fft.fftshift(np.fft.fftfreq(M, 1))
        
        # Plot
        fig, ax = plt.subplots(1,2)
        ax[0].imshow(F3, cmap = 'gray', interpolation = 'none', vmin = 0, vmax = 255)
        ax[0].set_title('Original')
        ax[1].imshow(y, cmap = 'gray', interpolation = 'none', vmin = 0, vmax = 255)
        ax[1].set_title('Filtered image')
        plt.savefig('C7F3.pdf', bbox_inches = 'tight',
            pad_inches = 0)
        plt.show()
        
        # Decibel frequency spectrum of image F3
        Y = 10*np.log10(np.abs(np.fft.fftshift(np.fft.fft2(F3))))
        
        
        # Plot
        fig, ax = plt.subplots(1,2)
        ax[0].imshow(Y, cmap = plt.cm.BuPu_r, aspect='auto', extent=(HfreqN.min(),HfreqN.max(),HfreqM.min(),HfreqM.max()))
        ax[0].set_title('Frequency spectrum of F3')
        ax[1].imshow(H, cmap = 'gray', aspect='auto', extent=(HfreqN.min(),HfreqN.max(),HfreqM.min(),HfreqM.max()))
        ax[1].set_title('Notch filter in frequency dom.')
        ax[0].set_xlim([-0.08,0.08])
        ax[0].set_ylim([-0.08,0.08])
        ax[1].set_xlim([-0.08,0.08])
        ax[1].set_ylim([-0.08,0.08])
        plt.savefig('C7F3freq.pdf', bbox_inches = 'tight',
            pad_inches = 0)
        plt.show()

    \end{lstlisting}
\caption{TASK C7-1}
\label{TASK C7-1}
\end{figure}



\begin{figure}[hbt!]
    \begin{lstlisting}
        """TASK C7-2"""



        import numpy as np 
        import matplotlib.pyplot as plt
        import os
        from PIL import Image
        import cv2
        from scipy.signal import convolve2d
        
        
        
        # Load images
        filedir = os.path.dirname(__file__)
        imagedir = 'images/'
        filename5 = 'F5.png'
        file5 = os.path.join(filedir, imagedir, filename5)
        
        filename3 = 'F3.png'
        file3 = os.path.join(filedir, imagedir, filename3)
        
        filename4 = 'F4.png'
        file4 = os.path.join(filedir, imagedir, filename4)
        
        filename5 = 'F5.png'
        file5 = os.path.join(filedir, imagedir, filename5)
        
        
        F3 = plt.imread(file3)
        F4 = plt.imread(file4)
        F5 = plt.imread(file5)
        
        
        # Transform to 0-255
        F3 = (F3-np.min(F3))/np.max(F3-np.min(F3))*255
        F4 = (F4-np.min(F4))/np.max(F4-np.min(F4))*255
        F5 = (F5-np.min(F5))/np.max(F5-np.min(F5))*255
        F3 = F3.astype('uint8')
        F4 = F4.astype('uint8')
        F5 = F5.astype('uint8')
        
        def butter_notch_filter(image, sigma, notches, n):
                """Butterworth notch reject, sigma defines the radius around the notch frequency
                and n defines the order.(how close to ideal you want)"""
                row, col = image.shape
                res = np.ones((row, col))
                for h in notches:
                    H = np.zeros((row, col))
                    for y in range(row):
                        for x in range(col):
                            # Notch
                            Dp = np.sqrt((y-int(row/2)-h[1])**2 + (x-int(col/2)-h[0])**2)
                            Dn = np.sqrt((y-int(row/2)+h[1])**2 + (x-int(col/2)+h[0])**2)
                            if Dp > 0 and Dn > 0:
                                H[y,x] = (1/(1+ (sigma/Dp)**n))*(1/(1+ (sigma/Dn)**n))
                            else:
                                H[y,x] = 0
                    # add product to new resulting image.
                    res *= H
                H = res
                X = np.fft.fftshift(np.fft.fft2(image))
                Y = np.fft.fftshift(X*H)
                y = np.fft.ifft2(Y)
                return np.abs(y), H
        
        def butterworth_hp_sharp(image, sigma, n, k):
                """Butterworth high pass filter, sigma defines the radius around the centered frequency
                and n defines the order.(how close to ideal you want)"""
                row, col = image.shape
                H = np.zeros((row, col))
                for y in range(row):
                    for x in range(col):
                        D = np.sqrt((y-int(row/2))**2 + (x-int(col/2))**2)
                        if D > 0:
                            H[y,x] = 1/(1+ (sigma/D)**(2*n))
                        else:
                            H[y,x] = 0
                X = np.fft.fftshift(np.fft.fft2(image))
                # sharp image
                Y = (1+k*H)*X
                Y = np.fft.fftshift(Y)
                y = np.fft.ifft2(Y)
                return np.abs(y)
        
        
        mid = 300 # Centered frequency, since notch filter have origo in center we need this to subtract.
        notches = np.array([[330-mid, 0]])
        
        
        # Use filters
        y, H = butter_notch_filter(F4, 1, notches, 3)
        y = butterworth_hp_sharp(y, 8, 5, k = 1)
        
        # Convert back to 255 after masking
        y = (y-np.min(y))/np.max(y-np.min(y))*255
        y = y.astype('uint8')
        
        # Get size of filter
        N , M = H.shape
        
        # Find real frequencies
        HfreqN = np.fft.fftshift(np.fft.fftfreq(N, 1))
        HfreqM = np.fft.fftshift(np.fft.fftfreq(M, 1))
        
        # Plot
        fig, ax = plt.subplots(1,2)
        ax[0].imshow(F4, cmap = 'gray')
        ax[0].set_title('Original')
        ax[1].imshow(y, cmap = 'gray')
        ax[1].set_title('Filtered image')
        plt.savefig('C7F4.pdf', bbox_inches = 'tight', pad_inches = 0)
        plt.show()
        
        # Decibel frequency spectrum of image F4
        Y = 10*np.log10(np.abs(np.fft.fftshift(np.fft.fft2(F4))))
        
        # Plot
        fig, ax = plt.subplots(1,2)
        ax[0].imshow(Y, cmap = plt.cm.BuPu_r, aspect='auto', extent=(HfreqN.min(),HfreqN.max(),HfreqM.min(),HfreqM.max()))
        ax[0].set_title('Frequency spectrum of F4')
        ax[1].imshow(H, cmap = 'gray', aspect='auto', extent=(HfreqN.min(),HfreqN.max(),HfreqM.min(),HfreqM.max()))
        ax[1].set_title('Notch filter in frequency dom.')
        ax[0].set_xlim([-0.08,0.08])
        ax[0].set_ylim([-0.08,0.08])
        ax[1].set_xlim([-0.08,0.08])
        ax[1].set_ylim([-0.08,0.08])
        plt.savefig('C7F4freq.pdf', bbox_inches = 'tight',
            pad_inches = 0)
        plt.show()
        
    \end{lstlisting}
\caption{TASK C7-2}
\label{TASK C7-2}
\end{figure}



\begin{figure}[hbt!]
    \begin{lstlisting}

        """TASK C7-3"""



        import numpy as np 
        import matplotlib.pyplot as plt
        import os
        from PIL import Image
        import cv2
        from scipy.signal import convolve2d
        
        
        
        # Loading images
        filedir = os.path.dirname(__file__)
        imagedir = 'images/'
        filename5 = 'F5.png'
        file5 = os.path.join(filedir, imagedir, filename5)
        
        filename3 = 'F3.png'
        file3 = os.path.join(filedir, imagedir, filename3)
        
        filename4 = 'F4.png'
        file4 = os.path.join(filedir, imagedir, filename4)
        
        filename5 = 'F5.png'
        file5 = os.path.join(filedir, imagedir, filename5)
        
        F3 = plt.imread(file3)
        F4 = plt.imread(file4)
        F5 = plt.imread(file5)
        
        
        # Transform to 0-255
        F3 = (F3-np.min(F3))/np.max(F3-np.min(F3))*255
        F4 = (F4-np.min(F4))/np.max(F4-np.min(F4))*255
        F5 = (F5-np.min(F5))/np.max(F5-np.min(F5))*255
        F3 = F3.astype('uint8')
        F4 = F4.astype('uint8')
        F5 = F5.astype('uint8')
        
        def butter_notch_filter(image, sigma, notches, n):
                """Butter notch filter"""
                row, col = image.shape
                res = np.ones((row, col))
                for h in notches:
                    H = np.zeros((row, col))
                    for y in range(row):
                        for x in range(col):
                            Dp = np.sqrt((y-int(row/2)-h[1])**2 + (x-int(col/2)-h[0])**2)
                            Dn = np.sqrt((y-int(row/2)+h[1])**2 + (x-int(col/2)+h[0])**2)
                            if Dp > 0 and Dn > 0:
                                H[y,x] = (1/(1+ (sigma/Dp)**n))*(1/(1+ (sigma/Dn)**n))
                            else:
                                H[y,x] = 0
                    res *= H
                H = res
                X = np.fft.fftshift(np.fft.fft2(image))
                Y = np.fft.fftshift(X*H)
                y = np.fft.ifft2(Y)
                return np.abs(y), H
        
        def butterworth_hp_sharp(image, sigma, n, k):
                """Butterworth high pass filter, sigma defines the radius around the centered frequency
                and n defines the order.(how close to ideal you want)"""
                row, col = image.shape
                H = np.zeros((row, col))
                for y in range(row):
                    for x in range(col):
                        D = np.sqrt((y-int(row/2))**2 + (x-int(col/2))**2)
                        if D > 0:
                            H[y,x] = 1/(1+ (sigma/D)**(2*n))
                        else:
                            H[y,x] = 0
                X = np.fft.fftshift(np.fft.fft2(image))
                # sharp image
                Y = (1+k*H)*X
                Y = np.fft.fftshift(Y)
                y = np.fft.ifft2(Y)
                return np.abs(y)
        
        
        # Centered frequency, since notch filter have origo in center we need this to subtract.
        mid = 300 
        # Notches to reject
        notches = np.array([[450-mid, 0], [330-mid, 0], [320-mid, 0], [315-mid, 0], [0, 450-mid], [0, 330-mid], [0, 320-mid], [0, 315-mid] ])
        
        # Use filters
        y, H = butter_notch_filter(F5, 1, notches, 3)
        y = butterworth_hp_sharp(y, 8, 5, k = 1)
        
        # transform values 0-255
        y = (y-np.min(y))/np.max(y-np.min(y))*255
        y = y.astype('uint8')
        
        # Get frequencies
        N , M = H.shape
        HfreqN = np.fft.fftshift(np.fft.fftfreq(N, 1))
        HfreqM = np.fft.fftshift(np.fft.fftfreq(M, 1))
        
        # Plot
        fig, ax = plt.subplots(1,2)
        ax[0].imshow(F5, cmap = 'gray', interpolation = 'none', vmin = 0, vmax = 255)
        ax[0].set_title('Original')
        ax[1].imshow(y, cmap = 'gray', interpolation = 'none', vmin = 0, vmax = 255)
        ax[1].set_title('Filtered image')
        plt.tight_layout()
        plt.savefig('C7F5.pdf', bbox_inches = 'tight',
            pad_inches = 0)
        plt.show()
        
        # Fourier of image
        Y = 10*np.log10(np.abs(np.fft.fftshift(np.fft.fft2(F5))))
        
        # plot
        fig, ax = plt.subplots(1,2)
        ax[0].imshow(Y, cmap=plt.cm.BuPu_r, aspect='auto', extent=(HfreqN.min(),HfreqN.max(),HfreqM.min(),HfreqM.max()))
        ax[0].set_title('Frequency spectrum of F5')
        ax[1].imshow(H*Y, cmap=plt.cm.BuPu_r, aspect='auto', extent=(HfreqN.min(),HfreqN.max(),HfreqM.min(),HfreqM.max()))
        ax[1].set_title('Frequency spectrum with notch applied.')
        ax[0].set_xlim([-0.08,0.08])
        ax[0].set_ylim([-0.08,0.08])
        ax[1].set_xlim([-0.08,0.08])
        ax[1].set_ylim([-0.08,0.08])
        plt.tight_layout()
        plt.savefig('removed_per_noise_freq.pdf', bbox_inches = 'tight',
            pad_inches = 0)
        plt.show()
    \end{lstlisting}
\caption{TASK C7-3}
\label{TASK C7-3}
\end{figure}



\section{References}
\begingroup
\renewcommand{\section}[2]{}%
%\renewcommand{\chapter}[2]{}% for other classes
\bibliographystyle{plainnat}
\bibliography{bibl}
\endgroup



% --------------------------------------------------------------
%     You don't have to mess with anything below this line.
% --------------------------------------------------------------
 





\end{document}



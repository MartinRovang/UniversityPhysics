% --------------------------------------------------------------
%     DENNE MALEN ER LAGET AV MARTIN SORIA RØVANG
%     TIL BRUK FOR OPPGAVELØSNINGER OG RAPPPRTER
%     GITHUB: github.com/martinrovang
% --------------------------------------------------------------


\documentclass[10pt]{article}
\usepackage{amsmath,amsthm,amssymb}
\usepackage{float}
\usepackage[table]{xcolor}
\usepackage[norsk]{babel}
\usepackage[utf8]{inputenc}
\UseRawInputEncoding
\usepackage{color}
\usepackage{graphicx}
\usepackage{listings}
\usepackage{natbib}
\usepackage{imakeidx}
\usepackage[a4paper]{geometry}
\usepackage[myheadings]{fullpage}
\usepackage{fancyhdr}
\usepackage{lastpage}
\usepackage{graphicx, wrapfig, subcaption, setspace, booktabs}
\usepackage[T1]{fontenc}
\usepackage[font=small, labelfont=bf]{caption}
\usepackage{fourier}
\usepackage[protrusion=true, expansion=true]{microtype}
\usepackage{url, lipsum}
\usepackage{tgbonum}
\usepackage{hyperref}
\usepackage{xcolor}
\usepackage[most]{tcolorbox}
\usepackage{mathtools}
\usepackage[page]{totalcount}
\usepackage{lastpage}


\newcommand{\HRule}[1]{\rule{\linewidth}{#1}}
\onehalfspacing
\setcounter{tocdepth}{5}
\setcounter{secnumdepth}{5}
\newcommand{\vect}[1]{\boldsymbol{#1}}

\definecolor{codegreen}{rgb}{0,0.6,0}
\definecolor{codegray}{rgb}{0.5,0.5,0.5}
\definecolor{codepurple}{rgb}{0.58,0,0.82}
\definecolor{backcolour}{rgb}{0.95,0.95,0.92}
\definecolor{skyblue}{rgb}{0.950, 1, 1}

\lstdefinestyle{mystyle}{
    backgroundcolor=\color{backcolour},   
    commentstyle=\color{codegreen},
    keywordstyle=\color{magenta},
    numberstyle=\tiny\color{codegray},
    stringstyle=\color{codepurple},
    basicstyle=\footnotesize,
    breakatwhitespace=false,         
    breaklines=true,                 
    captionpos=b,                    
    keepspaces=true,                 
    numbers=left,                    
    numbersep=5pt,                  
    showspaces=false,                
    showstringspaces=false,
    showtabs=false,                  
    tabsize=2,
    frame=single,
    %keywordstyle=\color{blue},
    language=Python,
    backgroundcolor = \color{skyblue}
}
 
\lstset{style=mystyle}
\lstset{
    basicstyle=\footnotesize\ttfamily,
  identifierstyle=\bfseries\color{green!40!black},
  commentstyle=\itshape\color{purple!40!black},
  keywordstyle=\color{blue},
  stringstyle=\color{orange},
}

\newcommand{\N}{\mathbb{N}}
\newcommand{\Z}{\mathbb{Z}}
 
\newenvironment{theorem}[2][Theorem]{\begin{trivlist}
\item[\hskip \labelsep {\bfseries #1}\hskip \labelsep {\bfseries #2.}]}{\end{trivlist}}
\newenvironment{lemma}[2][Lemma]{\begin{trivlist}
\item[\hskip \labelsep {\bfseries #1}\hskip \labelsep {\bfseries #2.}]}{\end{trivlist}}
\newenvironment{exercise}[2][Exercise]{\begin{trivlist}
\item[\hskip \labelsep {\bfseries #1}\hskip \labelsep {\bfseries #2.}]}{\end{trivlist}}
\newenvironment{problem}[2][Problem]{\begin{trivlist}
\item[\hskip \labelsep {\bfseries #1}\hskip \labelsep {\bfseries #2.}]}{\end{trivlist}}
\newenvironment{question}[2][Question]{\begin{trivlist}
\item[\hskip \labelsep {\bfseries #1}\hskip \labelsep {\bfseries #2.}]}{\end{trivlist}}
\newenvironment{corollary}[2][Corollary]{\begin{trivlist}
\item[\hskip \labelsep {\bfseries #1}\hskip \labelsep {\bfseries #2.}]}{\end{trivlist}}

\newenvironment{solution}{\begin{proof}[Solution]}{\end{proof}}
    
\makeindex[columns=3, title=Alphabetical Index, intoc]


% --------------------------------------------------------------
%                         Headers and footers
% --------------------------------------------------------------
\fancyhf{}
\pagestyle{fancy}
\rhead{Martin Soria R�vang}
\lhead{INF-1400-Objekt-Orientert programmering}
\rfoot{Side \thepage \, av \pageref{LastPage}}
\renewcommand{\headrulewidth}{0.3pt}
\usepackage{amssymb}
\usepackage{gensymb}
\usepackage{amsmath}
\usepackage{mathtools}

\begin{document}
% --------------------------------------------------------------
%                         FRONTPAGE
% --------------------------------------------------------------
% {\fontfamily{cmr}\selectfont
% \title{ \normalsize \textsc{}
% 		\\ [1.0cm] % How much upper margin
% 		%\HRule{0.5pt} \\
%         \LARGE \textbf{\uppercase{Obligatorisk Oppgave 3\\ Del II}
%         \HRule{0.5pt} \\ [0.5cm]
%         INF-1400-Objekt-Orientert programmering
%         %\HRule{2pt} \\ [0.5cm]
%         \\
% 		\normalsize \today \vspace*{5\baselineskip}}
% 		}

%         \date{}
% \author{
% 		Martin Soria R�vang \\ 
%         Universitetet i Troms� \\}

% % \begin{titlepage}
% \clearpage\maketitle
% \vspace{0.2\textheight}
% {\centering
% Inneholder \pageref{LastPage} \, sider, inkludert forside.\par
% }
% \thispagestyle{empty}
% % \end{titlepage}

% % \newpage
% % \tableofcontents
% \newpage
% --------------------------------------------------------------
%                         Start here
% --------------------------------------------------------------


\section{Part II}
\subsection{1}

En klasse er en slags bl�print for � lage objekter, dette kan ses p� som en fabrikk som lager biler, der objekter er bilene og fabrikkene er klassene.

    \begin{lstlisting}
class fabrikk:
        def __init__(self, wheelsize, color, motor):
            wheel.self = wheelsize
            color.self = color
            motor.self = motor
        def functions_that_does_stuff(self):
            ...

Bil1 = (50, 'green', 'RollsRoyce100X')
    \end{lstlisting}

I denne kodesnippeten over er et eksempel p� fabrikk/bil metaforet.




\subsection{2}

Arv(Inheritance) er at man $"$kopierer$"$ en annen klasse, og derfra endre p� akuratt de funksjonene/atributtene man �nsker. Dette kan brukes hvis man for eksempel skal ha en ny klasse som er ganske lik en annen, men man m� endre litt p� hvordan en/flere funksjon(er) fungerer. Syntaksen for dette er vist i figur(\ref{arv}).

\begin{figure}[hbt!]
    \begin{lstlisting}
    class Child(Parent):
        ....
    \end{lstlisting}
\caption{Child arver fra Parent}
\label{arv}
\end{figure}


\subsection{3}


\emph{Is-a} relasjon er det komme fra en klasse f.eks hvis man lager et objekt fra en klasse som heter \emph{hund} s� vil det objektet ha en \emph{is a dog} relasjon. \emph{has-a} relasjon er hva noe har, for eksempel ved bruk av hund-eksempelet s� kan hunden f.eks ha en funksjon som heter \emph{walk} s� vil klassen \emph{hund} ha \emph{has-a} relasjon med walk funkjonen, det samme vil gjelde objektene som blir laget fra denne klassen.


\subsection{4}

\emph{Encapsulation} er det � gjemme implementasjon bak grensesnittet i programmet ditt, i python kan man ikke blokkere noen ute av deler av koden(utenom � kompilere til .exe eller annet.) s� derfor har det blitt laget en konvensjon der man bruker \emph{\_.} f�r en funksjon/klasse/variabel som betyr at dette er \emph{privat} og dermed ikke b�r endres. Dette kan brukes hvis hvis disse har systemkritiske funksjoner som kan �delegge programmet hvis disse endres p�.

\subsection{5}

Polymorfisme er det at man har noe som endrer seg basert p� hva det skal gj�re. Hvis vi for eksempel har en klasse som er en gjenstand, og denne gjenstanden skal ha muligheten til � kun bevege seg diagonalt. Da vil man har en metode som heter \emph{move}. Videre vil man at en gjenstand ogs� skal kunne bevege seg, men denne ganger kun i en retning. Herfra kan man arve fra den f�rste klassen og endre p� move funksjonen slik at objekter fra dette kun beveger seg i en retning.
Her har man da polymorfisme fordi n� kan man f.eks kalle move p� objektene uten � tenke over hva de er for noe ogs� vil de bevege seg med forskjellige regler, et eksempel p� polymorfisme er \emph{duck typing}, ogs� kjent som $"$\emph{if it walks like a duck or swims like a duck, it's a duck}. I precoden har det blitt brukt vector2 fra pygame, her er det mest sannsynelig polymorfisme. I koden har vi polimorfisme mange steder. Ett eksempel er der vi har update-funksjon eller draw funksjon. Her bruker jeg kun en funksjon og det samme skjer (blir tegnet ut p� skjermen) uansett hva slags objekt det er (kule/romskip/vegg osv.)






% % --------------------------------------------------------------
% %     Reference og appendix
% % --------------------------------------------------------------
% \clearpage
% \newpage
% \section{Appendix}
% \section{Referanser}
% \begingroup
% \renewcommand{\section}[2]{}%
% %\renewcommand{\chapter}[2]{}% for other classes
% \bibliographystyle{plainnat}
% \bibliography{bibl}
% \endgroup



% --------------------------------------------------------------
%     You don't have to mess with anything below this line.
% --------------------------------------------------------------
 





\end{document}


